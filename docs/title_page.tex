\documentclass[3p, authoryear, review]{elsarticle}

\begin{document}


\begin{frontmatter}

  \title{Utility-Based Accessibility to Community Resources: An Application of Location-Based Services Data}
    \author[BYU]{Gregory Macfarlane%
  %
  \fnref{1}}
   \ead{gregmacfarlane@byu.edu} 
    \author[BYU]{Emma Stucki%
  %
  }
   \ead{stuckiemma@gmail.com} 
    \author[BYUPH]{Alisha Redelfs%
  %
  }
   \ead{alisha\_redelfs@byu.edu} 
    \author[BYUPH]{Lori Spruance%
  %
  }
   \ead{lori.spruance@byu.edu} 
      \affiliation[BYU]{Brigham Young University, Civil and Construction Engineering Department, 430 EB, Provo, Utah 84602}
    \affiliation[BYUPH]{Brigham Young University, Public Health Department, 4103 LSB, Provo, Utah 84602}
    \cortext[cor1]{Corresponding author}
    \fntext[1]{Corresponding Author}
  
  \begin{abstract}
  Understanding who in a community has access to its resources -- parks, libraries,
  grocery stores, etc. -- has profound equity implications, but typical methods
  to understand access to these resources are limited. Travel time buffers require
  researchers to assert mode of access as well as an arbitrary distance threshold;
  further, these methods do not distinguish between destination quality attributes
  in an effective way. In this research, we present a methodology to develop
  utility-based accessibility measures for parks, libraries, and grocery stores
  in Utah County, Utah. The method relies on passive location-based services data
  to model destination choice to these community resources; the destination choice
  model utility functions in turn allow us to develop a picture of regional access
  that is sensitive to: the quality and size of the destination resource;
  continuous (non-binary) travel impedance by multiple modes; and the
  sociodemographic attributes of the traveler. We then use this measure
  to explore equity in access to the specified community resources across
  income level in Utah County: the results reveal a discrepancy between which
  neighborhoods might be targeted for intervention using space-based analysis.
  \end{abstract}
    \begin{keyword}
    Accessibility \sep Passive Data \sep 
    Location Choice
  \end{keyword}
  
 \end{frontmatter}
 
 
\section*{Acknowledgements}

The authors are grateful to Brooke Jones, Kaeli Monahan, and Mali Smith for 
their help in gathering the grocery store and library information data. Connor 
Williams gathered the parks data, and Michael Copley assisted with the 
StreetLight data.This research did not receive any specific grant from funding 
agencies in the public, commercial, or not-for-profit sectors.

\end{document}